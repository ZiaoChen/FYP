\section{Background}
We are living in an era of information explosion. IBM reported that every day, 2.5 quintillion bytes of data are created --— so much that 90\% of the data in the world today have been created in the last two years alone \cite{IBMBigData}. As the amount of available data grows, the problem of managing the information becomes more difficult, which can lead to information overload.

Information overload occurs when individuals are exposed to more information than they can effectively utilize, which causes frustration and anxiety and leads to poor decision making \cite{Shenk1997}. According to a report from McKinsey, this is a serious issue faced by all types of upper-level management in the modern information economy \cite{Dean2011}. With so much overwhelming information available online, we often need uninterrupted time to synthesize information from many different sources to satisfy our needs.

Traditional search engines search documents for specified keywords and returns a list of the documents where the keywords are found, in the order of relevance. Very often, users wish to obtain a brief overview on the long list of results. However, current search engines do not summarize the search results for a query. Most of them neither create linkage among similar documents.

To meet users' growing information needs in today's data boom, we proposed an effective information retrieval solution with the use of concept ontology. A concept ontology, or ontology, is a specification of a shared vocabulary that can be used to model a domain of discourse. Ontology captures definitions and interrelationships of the entities that fundamentally exist for a particular domain \cite{Kapoor2010}. These entities are called \textit{concepts} for that particular domain.

By integrating information retrieval with concept ontology, users can filter search results for a query based on concepts and navigate among different documents effectively and efficiently. For this project, we developed a proof-of-concept information retrieval system in the domain of Digital Signal Processing (DSP).