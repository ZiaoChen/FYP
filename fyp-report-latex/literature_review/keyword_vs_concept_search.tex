\section{Keyword Search vs Concept Search}
Information retrieval is concerned with selecting some documents from a collection based on a user's information needs expressed using a query.

In the keyword-based approach, both documents and queries are represented as bags of lexical entities, namely keywords. A keyword can be a simple word or a compound word. As noted in \cite{Salton1988}, weights are associated with keywords to express their importance in the document (or the query). The weighting scheme is generally based on variations of the well-known \textit{tf*idf} formula.

In the concept-based approach, concepts are identified from the content of the document (or the query), and weighted according to both their frequency distribution and their semantic similarity to other concepts in the document (or the query) \cite{Boubekeur2013}.  


