\section{Searching}
\subsection{Searching and Filtering}
A search form module (\path{myfyp/dsp_search/forms.py}) was built to accept user query, pass the query to Solr, and retrieve search results. The module has a class named \texttt{SectionSearchForm}, which is inherited from Haystack's built-in class \texttt{SearchForm}.

The search form consists of a single field \texttt{q}, which is for query. Upon searching, the form will take the cleaned content of the \texttt{q} field. The searching process is done by Haystack via the \texttt{search()} method. The method returns a collection of documents, or a \textit{queryset} in Django's terms. Below is the code snippet. 

\begin{minted}{python}
def search(self):
    """ Override search() from parent class. """
    to_have = json.loads(self.data.get(self.with_field, '[]'))
    not_to_have = json.loads(self.data.get(self.without_field, '[]'))
    sqs = super(SectionSearchForm, self).search().models(Section)
    to_remove = self.get_removed_sids(sqs, to_have, not_to_have)
    return sqs.exclude(sid__in=to_remove)
\end{minted}

\texttt{to\char`_have} and \texttt{not\char`_to\char`_have} are two lists of concept IDs from the ontology. Each document in the search results must contain concepts specified in \texttt{to\char`_have} and must not contain concepts specified in \texttt{not\char`_to\char`_have}. \texttt{to\char`_remove} is a list of section IDs that indicate sections to be removed based on the two conditions enforced. 

The functionality of filtering search results is realized by passing \texttt{to\char`_have} and  \texttt{not\char`_to\char`_have} arguments via parameters in GET requests. The \texttt{search()} method first retrieves documents that match the query and then excludes those that are in the \texttt{to\char`_remove} list. The \texttt{to\char`_remove} list is generated based on \texttt{to\char`_have} and  \texttt{not\char`_to\char`_have} conditions.

\subsection{Search Results View}
A Django view is a Python function or a class that takes a Web request and returns a Web response. This response can be the HTML content of a Web page, a JSON document, a 404 error, or anything. The view itself contains logic that is necessary to return that response. The convention is to put views in files called \texttt{views.py} in Django's application directories.

A class named \texttt{SectionSearchView} was built to receive users' query request and return response back to users. \texttt{SectionSearchView} is dependent on the form class \texttt{SectionSearchForm} discussed earlier.

The \texttt{SectionSearchView} class includes the following instance methods:
\begin{itemize}
\item \texttt{get\char`_results\char`_count(self)}: Get the total number of sections that match the query;
\item \texttt{get\char`_section\char`_list(self)}: Get the list of IDs of sections that match the query;
\item \texttt{get\char`_concept\char`_tree(self)}: Get a JSON-formated concept tree that corresponds to the search results;
\item \texttt{get\char`_section\char`_count(self)}: Get a JSON-formatted dictionary describing the number of sections for each concept in the concept tree;
\item \texttt{get\char`_context\char`_data(self, *args, **kwargs)}: Get a dictionary representing the template context. 
\end{itemize}

Once receiving the queryset returned from \texttt{SectionSearchForm}'s \texttt{search()} method, \texttt{SectionSearchView} computes the number of search results via the \texttt{get\char`_results\char`_count()} method. It also  gets the concept tree of search results with \texttt{get\char`_concept\char`_tree()} method. More details on concept tree generation are discussed in Section \ref{subsec:concept_tree}. To indicate the number of sections that contain each concept in the concept tree, \texttt{get\char`_section\char`_count()} is then called. Lastly, the view \texttt{SectionSearchView} uses a Django template (\path{myfyp/dsp_search/results_page.html}) and the template context returned from \texttt{get\char`_context\char`_data()} to generate HTML response.


\subsection{Concept Tree Generation for Search Results}
\label{subsec:concept_tree}
Concept tree is a hierarchical tree structure that represents concepts related to the search results. In this thesis, \textit{immediate concepts} for a document refer to concepts that can be found in the \texttt{Concept mappings} table for the document. On the contrary, \textit{related concepts} for a document are those immediate concepts and their ancestor concepts. Thus, concept tree can also be thought as a set of related concepts for all documents in the search results.

The class \texttt{ConceptDictionaryGenerator}, located in \path{myfyp/dsp_search/views.py}, is used to generate the dictionarized concept tree as well as the section counts for each concept in the concept tree.

The class has an attribute named \texttt{section\char`_list}, which is a list of section IDs representing the search results. It has the following two methods called from the \texttt{SectionSearchView} class:
\begin{itemize}
\item \texttt{dictionarize\char`_concept\char`_hierarchy(self)}: Get a dictionarized concept tree based on the sections represented by \texttt{section\char`_list};
\item \texttt{get\char`_section\char`_counts(self)}: Get a dictionary describing the number of sections for each concept in the concept tree;
\end{itemize}

In a nut shell, generating concept tree is realized by:
\begin{enumerate}
\item Retrieve immediate concepts for each document in the search results. 
\item For each immediate concept, find the concept path from the root node to the current concept in the ontology.
\item Merge all concept paths into a concept tree, represented by JSON or a Python dictionary.
\end{enumerate}

Generating section counts for each concept in the concept tree is realized by:
\begin{enumerate}
\item Initialize counts for all related concepts to be 0.
\item For each document, retrieve all related concepts to the document and increment the counts for these concepts by 1.
\end{enumerate}

Python dictionary is the main data structure used to represent concept tree and concept path. There are instance methods in the class \texttt{ConceptDictionaryGenerator} for specific steps above.