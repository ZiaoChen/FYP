\section{Concept Ontology}
Ontologies have received more and more popularity in the area of knowledge management and knowledge sharing, especially after the evolution of the Semantic Web and its supporting technologies \cite{Berners-Lee2001}. \citet{Gruber1993} defined ontology as \enquote{a formal specification of a shared conceptualization}. This specification defines the concepts and the relationships among them, which are relevant for modeling a domain. 

Ontologies are becoming the cornerstone of the Semantic Web. Ontologies aim at capturing domain knowledge in a generic way and provide a commonly agreed understanding of a domain \cite{Kapoor2010}. The development of ontologies demands the use of various software tools called Ontology Editors. These tools \cite{OntologyEditor} like Prot\'eg\'e, SWOOP, and Neologism can be applied to different stages of the ontology life cycle including creation, implementation, and maintenance of ontologies. In \cite{Kapoor2010}, it mentions that developing ontology is an iterative process. The ontology is kept revised and refined from the rough first version after discussing with experts in the field. 

To construct an ontology one must have an ontology specification language. The Web Ontology Language (OWL) is the widely accepted Semantic Web language designed to represent rich and complex knowledge about things, groups of things, and relations among things \cite{OWL2012}. Knowledge expressed in OWL can be exploited by computer programs.
